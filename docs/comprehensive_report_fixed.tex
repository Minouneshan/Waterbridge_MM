\documentclass[12pt]{article}

% Encoding and font packages
\usepackage[utf8]{inputenc}
\usepackage[T1]{fontenc}

% Page layout
\usepackage[margin=1in]{geometry}

% Math and symbols
\usepackage{amsmath,amssymb}

% Graphics and figures
\usepackage{graphicx}

% Hyperlinks
\usepackage[colorlinks=true,linkcolor=blue,urlcolor=blue,citecolor=blue]{hyperref}

% For lists and tables
\usepackage{enumitem}
\usepackage{caption}

% Unicode character declarations to prevent compile errors
\DeclareUnicodeCharacter{2265}{\ensuremath{\geq}} % ≥
\DeclareUnicodeCharacter{2264}{\ensuremath{\leq}} % ≤
\DeclareUnicodeCharacter{2192}{\ensuremath{\rightarrow}} % →
\DeclareUnicodeCharacter{2013}{--} % – (en dash)
\DeclareUnicodeCharacter{2014}{---} % — (em dash)

% Declare additional Unicode characters used in tether citations and daggers
\DeclareUnicodeCharacter{3010}{[} % 【 maps to left bracket
\DeclareUnicodeCharacter{3011}{]} % 】 maps to right bracket
\DeclareUnicodeCharacter{2020}{\dagger} % † maps to dagger symbol

% Title
\title{Modern Mercantilism and Global Fragmentation (2025--2035)\\Enhanced Quantitative Analysis and Validation Framework}
\author{Mo Minoneshan\\
\small Email: \href{mailto:sm5943@columbia.edu}{sm5943@columbia.edu}\\
\small Repository: \url{github.com/Minouneshan/Waterbridge_MM}\\
\small Linkedin: \url{linkedin.com/in/minoneshan/}\\
\small Framework Version: 2.0 - Enhanced Econometric System
}
\date{August 2025}

\begin{document}

\maketitle

\tableofcontents

\clearpage

\section{Executive Summary}

Modern mercantilism describes the strategic use of state power to reshape global economic relationships through industrial policy, trade restrictions and technological competition.  Building on previous drafts and detailed modeling, this report presents a comprehensive and rigorous analysis of the forces driving global economic fragmentation and the probabilistic forecasts for key outcomes through 2035.  The core thesis is that great‑power security fears, domestic populist pressure and technological bifurcation are eroding liberal globalisation and giving rise to a durable regime of protectionism, state‑directed subsidies and parallel technology ecosystems.  Our analysis integrates quantitative data from multilateral institutions (WTO, IMF, UNCTAD), official statistics (U.S. Census, OECD), and industry reports with Bayesian belief networks, game‑theoretic reasoning and econometric models.  The resulting 25 binary forecasts are accompanied by detailed resolution criteria and confidence intervals, offering a roadmap for investors and policymakers navigating this fragmenting world.

Key findings include:

\begin{itemize}
\item \textbf{Institutional breakdown:}  There is a 90~\% probability that the WTO Appellate Body remains inoperative through 31 December 2027.  Reuters notes that the United States has blocked appointments since 2019, leaving the system inoperative[1], and the WTO confirms that there have been no active judges since November 2020[2].

\item \textbf{Tariff escalation:}  We assign a 70~\% probability that the global average most‑favoured‑nation (MFN) tariff rises by at least two percentage points above the 2022 baseline by 2026.  UNCTAD reports that about two‑thirds of global trade was tariff‑free in 2023 but the remaining trade faced high duties with manufacturing tariffs around 8 \%[3].  U.S. policy shifts have accelerated this trend: in April 2025 the United States imposed a 10 \% baseline tariff on nearly all imports and sharply raised duties on selected Chinese goods to over 100 \%[4].

\item \textbf{Supply‑chain reorientation:}  China's share of U.S. goods imports declined from 22 \% in 2017 to 17 \% in 2022[5], and friend‑shoring to Vietnam is accelerating: U.S. imports from Vietnam reached \$136.6 billion in 2024, up 19.3 \% from 2023[6].  Our regression and vector autoregression models project a 75~\% probability that China's share falls below 12 \% in at least one year between 2025–27 and a 72 \% probability that U.S. imports from Vietnam double their 2022 value by 2027.

\item \textbf{Intra‑Asian trade deepening:}  Asia is increasingly trading with itself: according to McKinsey, nearly 57 \% of Asian trade value originated within Asia in 2022[7], and the Asian Development Bank reports that intraregional trade comprised 58.5 \% of total Asian trade in 2020[8].  We assign a 65~\% probability that intra‑Asia merchandise trade reaches at least 35 \% of global trade by 2030.  This shift reflects the emergence of China‑centric supply networks and the Regional Comprehensive Economic Partnership (RCEP).

\item \textbf{Financial fragmentation with persistent dollar dominance:}  The U.S. dollar remains the dominant reserve currency, comprising about 59 \% of global reserves[9] despite declining from over 70 \% in 2000[9].  The Chinese renminbi accounts for only about 2.3 \% of reserves[10].  We project a 77 \% probability that the dollar's share remains above 50 \% in 2030 and a 72 \% probability that the RMB stays below 10 \% of reserves and under 5 \% of SWIFT payments.

\item \textbf{Carbon‑tariff proliferation:}  The European Union's Carbon Border Adjustment Mechanism (CBAM) moves from a reporting‑only transitional phase (2023–25) to a definitive regime in 2026, requiring importers of carbon‑intensive products to purchase emission certificates[11].  We assign a 78 \% probability that the CBAM is fully operational by 2026 with high compliance and that at least three other G‑20 economies adopt comparable carbon tariffs by 2030.

\item \textbf{Rising global defence spending:}  Heightened geopolitical tensions and proxy conflicts have pushed world military expenditure to \$2.72 trillion in 2024, up 9.4 \% from 2023, the steepest annual increase since the Cold War[12].  We estimate a 65 \% probability that global defence spending will rise at least 25 \% in real terms over 2025–2030.
\end{itemize}

These headline forecasts are part of a coherent framework that models interdependencies among trade policy, supply chains, technology standards, finance, climate policy and security.  The sections that follow detail the underlying drivers, actor strategies, methodology, forecast interdependencies and deeper analytical insights drawn from earlier drafts and technical appendices.

\clearpage

\section{Binary Forecasts and Resolution Criteria}

The 25 forecasts summarise the key events that underpin our modern‑mercantilist outlook.  Each item lists the forecast ID, a concise statement, its probability and the resolution criteria used to score outcomes.  Probabilities reflect our posterior beliefs after evidence integration, and the resolution criteria anchor the forecasts to objective data sources.

\begin{description}[style=nextline]
\item[\textbf{F1 – Average applied MFN tariff rises \(\geq 2\) percentage points relative to the 2022 baseline by 2026} (70 \%).] \textit{Resolution:} WTO World Tariff Profiles 2027 vs 2023; "yes" if the global average MFN rate is at least two percentage points higher than the 2022 baseline (4.1 \%).

\item[\textbf{F2 – WTO Appellate Body remains inoperative on 31 December 2027} (80 \%).] \textit{Resolution:} WTO Secretariat roster of Appellate Body members; "yes" if there are zero active members at year‑end.

\item[\textbf{F3 – World trade volume growth is slower than world GDP growth in every year 2025–27} (75 \%).] \textit{Resolution:} IMF World Economic Outlook (Oct 2028 edition); "yes" if merchandise trade volume grows more slowly than GDP in each year.

\item[\textbf{F4 – At least three of the top‑ten grain exporters impose new export bans by end‑2026} (65 \%).] \textit{Resolution:} IFPRI Food Trade Policy Tracker and official gazettes; "yes" if three distinct top‑ten grain exporters enact new bans by end‑2026.

\item[\textbf{F5 – Global trade in services (\% GDP) exceeds 2020 levels while goods trade remains at or below 2020 levels through 2030} (80 \%).] \textit{Resolution:} WTO Trade in Services database; "yes" if by 2030 services trade (\% GDP) exceeds the 2020 share (5.4 \%) and goods trade (\% GDP) is at or below the 2020 baseline (19 \%).

\item[\textbf{F6 – China's share of U.S. goods imports falls below 12 \% in at least one year between 2025–27} (75 \%).] \textit{Resolution:} U.S. Census Bureau FT‑900 (Feb 2028 release); "yes" if China's share falls below 12 \% in any year 2025–27 (current trend: 13.9 \% in 2024).

\item[\textbf{F7 – U.S. imports from Vietnam double their 2022 value (\$127.5 billion) by 2027} (72 \%).] \textit{Resolution:} U.S. International Trade Commission DataWeb; "yes" if 2027 imports \(\geq\) \$255 billion (double the 2022 baseline of \$127.5 billion).

\item[\textbf{F8 – Intra‑Asia (China‑centric) trade comprises at least 35 \% of world merchandise trade by 2030} (65 \%).] \textit{Resolution:} UN Comtrade database; "yes" if intraregional Asian trade accounts for \(\geq 35\) \% of global trade by 2030 (current: approximately 32 \%).

\item[\textbf{F9 – More than 50 \% of China's exports go to the Global South by 2030} (68 \%).] \textit{Resolution:} China Customs Yearbook 2031; "yes" if exports to the Global South exceed 50 \% of total Chinese exports (current: roughly 47 \%).

\item[\textbf{F10 – BRICS share of world PPP‑GDP reaches \(\geq 40\,\%\) and BRICS Bank capital reaches \(\geq \$200\,\text{billion}\) by 2030} (70 \%).] \textit{Resolution:} IMF WEO 2031 and New Development Bank Annual Report; "yes" if both conditions hold simultaneously (current BRICS share: 36.2 \%; NDB capital: \$100 billion).

\item[\textbf{F11 – The European Union enacts EUR 100 billion or more in new "strategic autonomy" subsidies by 2028 and no member exits the EU} (78 \%).] \textit{Resolution:} European Commission State‑Aid Scoreboard 2029 and EU Council records; "yes" if cumulative new subsidies reach the threshold and no member state withdraws.

\item[\textbf{F12 – At least five G‑20 economies announce \(\geq \$50\) billion each in industrial subsidies by 2026} (77 \%).] \textit{Resolution:} Official government budgets or legislation (tracked via IMF policy trackers); "yes" if five distinct G‑20 members announce industrial subsidy packages meeting the threshold.

\item[\textbf{F13 – At least two U.S. fabs below 5 nm begin volume production by 2027} (62 \%).] \textit{Resolution:} Semiconductor Industry Association (SIA) State of the U.S. Industry 2028; "yes" if two or more sub‑5 nm fabrication facilities achieve commercial volume production.

\item[\textbf{F14 – Distinct U.S.-led vs. China‑led tech standards dominate at least five verticals by 2027} (85 \%).] \textit{Resolution:} ISO/ITU standards catalogues and industry reports (Q4 2027); "yes" if clear bifurcation exists in three or more technology verticals (e.g., 5G, AI, EVs, semiconductors).

\item[\textbf{F15 – China produces \(\geq 70\,\%\) of its \(\geq 28\) nm chips domestically in 2030, but <30 \% of its <5 nm chips} (58 \%).] \textit{Resolution:} IC Insights McClean Report 2031; "yes" if both conditions are satisfied simultaneously: domestic production \(\geq 70\,\%\) for mature nodes and <30 \% for advanced nodes.

\item[\textbf{F16 – China expands export controls to at least one additional critical mineral by end‑2025} (60 \%).] \textit{Resolution:} Ministry of Commerce official notices; "yes" if new minerals are added to export control lists beyond graphite, gallium and germanium.

\item[\textbf{F17 – U.S. dollar share of global foreign‑exchange reserves remains above 55.5 \% on 30 June 2030} (66 \%).] \textit{Resolution:} IMF COFER report for Q2 2030; "yes" if the USD reserve share exceeds 55.5 \%.

\item[\textbf{F18 – Renminbi share of global reserves stays below 3 \% in 2030 and <5 \% of SWIFT payments} (62 \%).] \textit{Resolution:} IMF COFER 2030 and SWIFT RMB Tracker (Dec 2030); "yes" if both conditions are met (current: approximately 2.7 \% of reserves and 2.3 \% of SWIFT payments).

\item[\textbf{F19 – At least three of the top‑ten oil exporters regularly price \(\geq 20\,\%\) of their exports in non‑USD currencies by 2030} (58 \%).] \textit{Resolution:} Oxford Institute for Energy Studies Currency of Commodity Trade Survey 2031; "yes" if three or more exporters meet the non‑USD pricing threshold.

\item[\textbf{F20 – EU CBAM fully operational by 2026 and at least seven other G‑20 carbon tariffs by 2029} (64 \%).] \textit{Resolution:} Official government sources (EU Official Journal and national legislation); "yes" if the EU Carbon Border Adjustment Mechanism is operational and three additional G‑20 carbon tariffs are implemented.

\item[\textbf{F21 – Middle‑income countries see inflation \(\geq 8\,\%\) in at least two years between 2025–28} (75 \%).] \textit{Resolution:} IMF World Economic Outlook Database (April 2029); "yes" if weighted‑average inflation in middle‑income economies exceeds 8 \% in two or more years during the period.

\item[\textbf{F22 – At least five countries establish sovereign wealth funds with \(\geq \$50\) billion assets under management by 2030} (68 \%).] \textit{Resolution:} Sovereign Wealth Fund Institute Annual Report 2031; "yes" if five or more countries that do not currently operate SWFs establish funds meeting the AUM threshold.

\item[\textbf{F23 – China's AI researcher workforce exceeds 50,000 while U.S. workforce remains <30,000 by 2030} (70 \%).] \textit{Resolution:} National Science Foundation \textit{Science & Engineering Indicators 2031} and Chinese Ministry of Science and Technology Annual Report; "yes" if both conditions hold simultaneously.

\item[\textbf{F24 – At least three G‑20 countries implement comprehensive data localisation laws by 2027} (80 \%).] \textit{Resolution:} OECD \textit{Digital Economy Outlook 2028}; "yes" if three or more economies enact laws requiring domestic storage/processing of citizen data across multiple sectors.

\item[\textbf{F25 – Global defence spending increases by at least 25 \% in real terms 2025–30} (65 \%).] \textit{Resolution:} SIPRI Military Expenditure Database 2031 (constant 2024 USD); "yes" if aggregate global defence spending rises at least 25 \% above the 2024 baseline by 2030.
\end{description}

\end{document}
